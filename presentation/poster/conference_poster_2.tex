%%%%%%%%%%%%%%%%%%%%%%%%%%%%%%%%%%%%%%%%%
% Dreuw & Deselaer's Poster
% LaTeX Template
% Version 1.0 (11/04/13)
%
% Created by:
% Philippe Dreuw and Thomas Deselaers
% http://www-i6.informatik.rwth-aachen.de/~dreuw/latexbeamerposter.php
%
% This template has been downloaded from:
% http://www.LaTeXTemplates.com
%
% License:
% CC BY-NC-SA 3.0 (http://creativecommons.org/licenses/by-nc-sa/3.0/)
%
%%%%%%%%%%%%%%%%%%%%%%%%%%%%%%%%%%%%%%%%%

%----------------------------------------------------------------------------------------
%	PACKAGES AND OTHER DOCUMENT CONFIGURATIONS
%----------------------------------------------------------------------------------------

\documentclass[final,hyperref={pdfpagelabels=false}]{beamer}

\usepackage[orientation=portrait,size=a0,scale=1.4]{beamerposter} % Use the beamerposter package for laying out the poster with a portrait orientation and an a0 paper size

\usetheme{I6pd2} % Use the I6pd2 theme supplied with this template

\usepackage[english]{babel} % English language/hyphenation

\usepackage{amsmath,amsthm,amssymb,latexsym} % For including math equations, theorems, symbols, etc

%\usepackage{times}\usefonttheme{professionalfonts}  % Uncomment to use Times as the main font
%\usefonttheme[onlymath]{serif} % Uncomment to use a Serif font within math environments

\boldmath % Use bold for everything within the math environment

\usepackage{booktabs} % Top and bottom rules for tables

\graphicspath{{figures/}} % Location of the graphics files

\usecaptiontemplate{\small\structure{\insertcaptionname~\insertcaptionnumber: }\insertcaption} % A fix for figure numbering

%----------------------------------------------------------------------------------------
%	TITLE SECTION 
%----------------------------------------------------------------------------------------

\title{\huge ESS-NW/CAR} % Poster title

\author{Leon Fernandez, Jonas Ekman, Fredrik Hyyrynen, Jacob Kimblad, Yini Gao and  Yifan Ruan} % Author(s)

\institute{ MF2063 Embedded Systems Design Project} % Institution(s)

%----------------------------------------------------------------------------------------
%	FOOTER TEXT
%----------------------------------------------------------------------------------------

\newcommand{\leftfoot}{ESS-NW/CAR} % Left footer text

\newcommand{\rightfoot}{MF2063} % Right footer text

%----------------------------------------------------------------------------------------

\begin{document}

\addtobeamertemplate{block end}{}{\vspace*{2ex}} % White space under blocks

\begin{frame}[t] % The whole poster is enclosed in one beamer frame

\begin{columns}[t] % The whole poster consists of two major columns, each of which can be subdivided further with another \begin{columns} block - the [t] argument aligns each column's content to the top

\begin{column}{.02\textwidth}\end{column} % Empty spacer column

\begin{column}{.465\textwidth} % The first column


%----------------------------------------------------------------------------------------
%	INTRODUCTION
%----------------------------------------------------------------------------------------
            
\begin{block}{Introduction}

\begin{itemize}
\item This project is to create an autonomous car that is based on an software-defined network (SDN). On the car is different sensors placed to monitor the performance of the car and all this information is used in a control unit to control the car. The information providers and the consumers have to be connected and this is done via an SDN network there a control unit decides how the packet should be sent via the network. The information providers in the car is ultrasound to measure distance, object recognition and speed sensor. 
\end{itemize}

\end{block}

%----------------------------------------------------------------------------------------
%	NETWORK
%----------------------------------------------------------------------------------------

\begin{block}{Network}
    \begin{itemize}
        \item The network topology in the car is as shown in figure \ref{pic:network}, there all the information providers and consumers are connected via an SDN network.  The controller for the SDN is a raspberry pi model 3, and runs an SDN controller program called Ryu and is used to control the network on how the packages should travel and what priority they have. 
        \item VSOMEIP 
    \end{itemize}

    \begin{figure}
        \includegraphics[width=1\linewidth]{network.png}
        \caption{Network topology}
        \label{pic:network}
        \end{figure}

\end{block}

%----------------------------------------------------------------------------------------
%	Control and Sensors 
%----------------------------------------------------------------------------------------

\begin{block}{Control and sensors}

\begin{itemize}
    \item ultrasound
    \item Object recognition
    \item Speed sensor
\end{itemize}

\end{block}

%----------------------------------------------------------------------------------------
%	
%----------------------------------------------------------------------------------------

\begin{block}{Mathematical Section}

\begin{itemize}
\item Maecenas Ultricies Feugiat Velit Non Mattis.
\begin{itemize}
\item Duis ante erat, bibendum nec tempus nec, interdum quis est. Nulla at mollis tortor. Phasellus quis leo dolor, aliquam laoreet orci $X$ Donec dapibus sagittis neque eu nec, interdum quis est. $Y_n, n=1,\cdots,N$ ndum nec tempus nec, interd
\begin{align*}
X \rightarrow r(X) & = \arg \max_{c} \Big\{ \max_n \big\{ \sum_{x_i \in X} \delta(x_i,Y_{n,c})\big\} \Big\} 
\end{align*}
\item Cras faucibus scelerisque cursus. Proin ut vestibulum augue. $\delta(x_i,Y_{n,c})$
\end{itemize}
\item Fusce tempus arcu id ligula varius dictum. Donec ut nisl dui, ac consectetur elit. In nec enim porta augue venenatis sollicitudin. Phasellus quis nunc neque. Suspendisse mauris diam, suscipit non gravida in, placerat id enim. Ut nec ipsum in lectus ultrices sagittis.
\end{itemize}

\end{block}

%----------------------------------------------------------------------------------------

\end{column} % End of the first column

\begin{column}{.03\textwidth}\end{column} % Empty spacer column
 
\begin{column}{.465\textwidth} % The second column

%----------------------------------------------------------------------------------------
%	RESULTS
%----------------------------------------------------------------------------------------

\begin{block}{Results: Table}

\begin{itemize}
\item Ased Aliquet Luctus Lectus
\end{itemize}

\begin{table}
\begin{tabular}{l l l}
\toprule
\textbf{Treatments} & \textbf{Response 1} & \textbf{Response 2}\\
\midrule
Treatment 1 & 0.0003262 & 0.562 \\
Treatment 2 & 0.0015681 & 0.910 \\
Treatment 3 & 0.0009271 & 0.296 \\
\bottomrule
\end{tabular}
\caption{Table caption}
\end{table}

\begin{itemize}
\item Sollicitudin Vel Orci
\item Maecenas Ultricies Feugiat Velit Non Mattis.
\end{itemize}

\begin{table}
\begin{tabular}{l l l}
\toprule
\textbf{Treatments} & \textbf{Response 1} & \textbf{Response 2}\\
\midrule
Treatment 1 & 0.0003262 & 0.562 \\
Treatment 2 & 0.0015681 & 0.910 \\
Treatment 3 & 0.0009271 & 0.296 \\
\bottomrule
\end{tabular}
\caption{Table caption}
\end{table}
     
\end{block}

%------------------------------------------------

\begin{block}{Results: Figure}

\begin{figure}
\includegraphics[width=0.8\linewidth]{placeholder.jpg}
\caption{Figure caption}
\end{figure}

\end{block}

%----------------------------------------------------------------------------------------
%	CONCLUSION
%----------------------------------------------------------------------------------------

\begin{block}{Conclusion}

\begin{itemize}
\item Opet volutpat ligula. Duis semper lorem eget dui dignissim porttitor. Nulla facilisi. In ullamcorper lorem quis dolor iaculis nec egestas enim ultricies. Cras ut mauris elit, ut lacinia dui. Proin in ante et libero hendrerit iaculis.
\item Nulla eu erat a urna laoreet auctor id a turpis. Nam mollis tristique neque eu luctus. Suspendisse rutrum congue nisi sed convallis. 
\item Aenean id neque dolor.
\item Opet volutpat ligula. Duis semper lorem eget dui dignissim porttitor. Nulla facilisi. In ullamcorper lorem quis dolor iaculis nec egestas enim ultricies. Cras ut mauris elit, ut lacinia dui. Proin in ante et libero hendrerit iaculis.
\end{itemize}

\end{block}

%----------------------------------------------------------------------------------------
%	ACKNOWLEDGEMENTS
%----------------------------------------------------------------------------------------

\begin{block}{Acknowledgments}

\begin{itemize}
\item We want to thank the project owners for their support they have given us under this project. 
\end{itemize}

\end{block}

%----------------------------------------------------------------------------------------
%	CONTACT INFORMATION
%----------------------------------------------------------------------------------------

\setbeamercolor{block title}{fg=black,bg=orange!70} % Change the block title color

\begin{block}{Prodject Information}

\begin{itemize}
\item Github: \href{https://github.com/fhyy/MF2063-ESS-NW-CAR}{https://github.com/fhyy/MF2063-ESS-NW-CAR}

\end{itemize}

\end{block}

%----------------------------------------------------------------------------------------

\end{column} % End of the second column

\begin{column}{.015\textwidth}\end{column} % Empty spacer column

\end{columns} % End of all the columns in the poster

\end{frame} % End of the enclosing frame

\end{document}