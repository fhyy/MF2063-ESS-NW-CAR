% !TEX TS-program = pdflatex
% !TEX encoding = UTF-8 Unicode

% This is a simple template for a LaTeX document using the "article" class.
% See "book", "report", "letter" for other types of document.

\documentclass[11pt, titlepage]{article} % use larger type; default would be 10pt

\usepackage[utf8]{inputenc} % set input encoding (not needed with XeLaTeX)

%%% Examples of Article customizations
% These packages are optional, depending whether you want the features they provide.
% See the LaTeX Companion or other references for full information.

%%% PAGE DIMENSIONS
\usepackage{geometry} % to change the page dimensions
\geometry{a4paper} % or letterpaper (US) or a5paper or....
\geometry{margin=2cm, headsep=5mm, includefoot, includehead}

\usepackage{graphicx} % support the \includegraphics command and options
\graphicspath{{figures/}} % Location of the graphics files
% \usepackage[parfill]{parskip} % Activate to begin paragraphs with an empty line rather than an indent

%%% PACKAGES
\usepackage{booktabs} % for much better looking tables
\usepackage{array} % for better arrays (eg matrices) in maths
\usepackage{paralist} % very flexible & customisable lists (eg. enumerate/itemize, etc.)
\usepackage{verbatim} % adds environment for commenting out blocks of text & for better verbatim
\usepackage{subfig} % make it possible to include more than one captioned figure/table in a single float
% These packages are all incorporated in the memoir class to one degree or another...

%%% HEADERS & FOOTERS
\usepackage{fancyhdr} % This should be set AFTER setting up the page geometry
\pagestyle{fancy} % options: empty , plain , fancy
\fancyhead{}
%%% SECTION TITLE APPEARANCE
\usepackage{sectsty}
\allsectionsfont{\sffamily\mdseries\upshape} % (See the fntguide.pdf for font help)
% (This matches ConTeXt defaults)

%%% ToC (table of contents) APPEARANCE
\usepackage[nottoc,notlof,notlot]{tocbibind} % Put the bibliography in the ToC
\usepackage[titles,subfigure]{tocloft} % Alter the style of the Table of Contents
\renewcommand{\cftsecfont}{\rmfamily\mdseries\upshape}
\renewcommand{\cftsecpagefont}{\rmfamily\mdseries\upshape} % No bold!

\usepackage{lastpage}
\usepackage{rotating}
%---------- Enable IEEEtran.bst configurations ------
\usepackage{IEEEtrantools}
%----------------------------------------------------

%%% END Article customizations

%%% The "real" document content comes below...

%%% Header %%%%%%%%%%%%%%%%%%%%%%%%%%%%%
\setlength{\headheight}{53pt}
\lhead{MF2063 Embedded Systems Design Project \\		 
       ESS-CAR/ESS-NW \\
       Leon Fernandez, leonfe@kth.se}
\rhead{Final Report \\
       Version 1 \\
       \thepage(\pageref{LastPage})}
\renewcommand{\headrulewidth}{1pt}
%%%%%%%%%%%%%%%%%%%%%%%%%%%%%%%%%%%%%%%%

%%% Footer %%%%%%%%%%%%%%%%%%%%%%%%%%%%%
%\cfoot{blablablabla}
%\renewcommand{\footrulewidth}{0.4pt}
%%%%%%%%%%%%%%%%%%%%%%%%%%%%%%%%%%%%%%%%
%\date{} % Activate to display a given date or no date (if empty),
         % otherwise the current date is printed 

\begin{document}
%\maketitle
\bstctlcite{BSTcontrol} % IEEEtran.bst controls enabled


%----------------------------------------------------------------------------------------
%	TITLE PAGE
%----------------------------------------------------------------------------------------

\begin{titlepage} % Suppresses displaying the page number on the title page and the subsequent page counts as page 1
	\newcommand{\HRule}{\rule{\linewidth}{0.5mm}} % Defines a new command for horizontal lines, change thickness here
	
	\center % Centre everything on the page
	
	%------------------------------------------------
	%	Headings
	%------------------------------------------------
	
	\begin{figure}
   		\centering
    	\includegraphics[scale=1]{kthLogo.png}
	\end{figure}
	
	\textsc{\LARGE KTH Mechatronics Advanced Course}\\[1cm] % Main heading such as the name of your university/college
	
	\textsc{\Large MF2063, HT 2018}\\[0.5cm] % Major heading such as course name
	
	\textsc{\Large FINAL REPORT}\\[0.5cm] % Major heading as well
	
	%------------------------------------------------
	%	Title
	%------------------------------------------------
	
	\HRule\\[0.4cm]
	
	{\huge\bfseries ESS-NW/ESS-CAR}\\[0.4cm] % Title of your document
	
	\HRule\\[1.5cm]	
	
	%------------------------------------------------
	%	Author(s)
	%------------------------------------------------
	
	\begin{minipage}{0.4\textwidth}
		\begin{flushleft}
			\large
                        \textsc{Jonas Ekman}
			\\
			\textsc{Yini Gao}
                        \\
                        \textsc{Jacob Kimblad}
		\end{flushleft}
	\end{minipage}
	~
	\begin{minipage}{0.4\textwidth}
		\begin{flushright}
			\large
                        \textsc{Leon Fernandez}
			\\
			\textsc{Fredrik Hyyrynen}
                        \\
                        \textsc{Yifan Ruan}
		\end{flushright}
	\end{minipage}
	
	% If you don't want a supervisor, uncomment the two lines below
        % and comment the code above
	%{\large\textit{Author}}\\
	%John \textsc{Smith} % Your name
	
	%------------------------------------------------
	%	Date
	%------------------------------------------------
	
	\vfill\vfill\vfill % Position the date 3/4 down the remaining page
	
	{\large\today} % Date, change the \today to a set date if you want
                       % to be precise
	
	%------------------------------------------------
	%	Logo
	%------------------------------------------------
	
	%\vfill\vfill
	%\includegraphics[width=0.2\textwidth]{placeholder.jpg}\\[1cm]
        % Include a department/university logo
        % - this will require the graphicx package
	 
	%-------------------------------------------------------------------
	
	\vfill % Push the date up 1/4 of the remaining page
	
\end{titlepage}

%-------------------------------------------------------------------------

\clearpage
\section*{Abstract}
Abstract starts here,
what should be included:

The problem issue subject being addressed

How the problem is tackled

Overview of the results, and indication as to what level they solve the problem.

Implications of the results
\clearpage
%-------------------------------------------------------------------------
\clearpage
\tableofcontents
\clearpage
%-------------------------------------------------------------------------
\clearpage
\listoffigures
\clearpage
%-------------------------------------------------------------------------
\clearpage
\listoftables
\clearpage
%-------------------------------------------------------------------------
\clearpage
\section{Introduction}
This report presents the process and results of two projects "Embedded Service for Self-adaptive Network" (ESS-NW) and "Embedded Service for Self-adaptive Car" (ESS-CAR). This chapter will start by describing the background of the two projects. The next thing to be described is formulation, goals and motivation of the two projects. Following this will be a short discussion on the delimitations for our team. The last part of this chapter will present an explicit report disposition which helps readers to get a sense of the overall report.

\subsection{Background}

\subsubsection{Background subsection blabla}

\subsection{Project Description}

\subsubsection{Project Description sub blabla}

\subsection{Delimitations}

\subsection{Report disposition}

%-------------------------------------------------------------------------
\clearpage
\section{Literature Review and State of the Art}

%-------------------------------------------------------------------------
\clearpage
\section{Methodology}
\subsection{Engineering approaches ?}
\subsection{Tool-chains ?}
\subsection{Project management}
Scrum project management is used during the process of our projects.

%-------------------------------------------------------------------------
\clearpage
\section{Implementation}
\subsection{System overview}
maybe put communication diagram here
\subsection{Implementing SDN network}

\subsection{Communication between Beaglebone and Arduino ?}

\subsection{Sensors}
\subsubsection{Ultrasonic sensor}
\subsubsection{Reflective object sensor}
\subsubsection{Camera}

\subsection{Controlling actuators}
\subsubsection{Steering servo}
\subsubsection{Motor ESC}
\subsection{Assemble the car, power supply, etc}

%-------------------------------------------------------------------------
\clearpage
\section{Verification and Validation}

%-------------------------------------------------------------------------
\clearpage
\section{Results}

%-------------------------------------------------------------------------
\clearpage
\section{Discussion and Conclusion}

%-------------------------------------------------------------------------
\clearpage
\section{Future Work}

\clearpage
%TODO reference
\clearpage

\clearpage
%TODO appendix
\appendix

\clearpage

\end{document}
