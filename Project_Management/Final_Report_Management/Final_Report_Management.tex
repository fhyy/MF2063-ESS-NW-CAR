% !TEX TS-program = pdflatex
% !TEX encoding = UTF-8 Unicode

% This is a simple template for a LaTeX document using the "article" class.
% See "book", "report", "letter" for other types of document.

\documentclass[11pt, titlepage]{article} % use larger type; default would be 10pt

\usepackage[utf8]{inputenc} % set input encoding (not needed with XeLaTeX)

%%% Examples of Article customizations
% These packages are optional, depending whether you want the features they provide.
% See the LaTeX Companion or other references for full information.

%%% PAGE DIMENSIONS
\usepackage{geometry} % to change the page dimensions
\geometry{a4paper} % or letterpaper (US) or a5paper or....
\geometry{margin=2cm, headsep=5mm, includefoot, includehead}

\usepackage{graphicx} % support the \includegraphics command and options
\graphicspath{{figures/}} % Location of the graphics files
% \usepackage[parfill]{parskip} % Activate to begin paragraphs with an empty line rather than an indent

%%% PACKAGES
\usepackage{booktabs} % for much better looking tables
\usepackage{array} % for better arrays (eg matrices) in maths
\usepackage{paralist} % very flexible & customisable lists (eg. enumerate/itemize, etc.)
\usepackage{verbatim} % adds environment for commenting out blocks of text & for better verbatim
\usepackage{subfig} % make it possible to include more than one captioned figure/table in a single float
% These packages are all incorporated in the memoir class to one degree or another...

%%% HEADERS & FOOTERS
\usepackage{fancyhdr} % This should be set AFTER setting up the page geometry
\pagestyle{fancy} % options: empty , plain , fancy
\fancyhead{}
%%% SECTION TITLE APPEARANCE
\usepackage{sectsty}
\allsectionsfont{\sffamily\mdseries\upshape} % (See the fntguide.pdf for font help)
% (This matches ConTeXt defaults)

%%% ToC (table of contents) APPEARANCE
\usepackage[nottoc,notlof,notlot]{tocbibind} % Put the bibliography in the ToC
\usepackage[titles,subfigure]{tocloft} % Alter the style of the Table of Contents
\renewcommand{\cftsecfont}{\rmfamily\mdseries\upshape}
\renewcommand{\cftsecpagefont}{\rmfamily\mdseries\upshape} % No bold!

\usepackage{lastpage}
\usepackage{rotating}
\usepackage{textcomp} %get the correct micro sec display
\usepackage{float}
%---------- Enable IEEEtran.bst configurations ------
\usepackage{IEEEtrantools}
\usepackage{verbatim}
%----------------------------------------------------

%%% END Article customizations

%%% The "real" document content comes below...

%%% Header %%%%%%%%%%%%%%%%%%%%%%%%%%%%%
\setlength{\headheight}{53pt}
\lhead{EH2760 Management of Projects \\		 
       ESS-NW/ESS-CAR \\
       Leon Fernandez, leonfe@kth.se}
\rhead{Final Report \\
       Version 1 \\
       \thepage(\pageref{LastPage})}
\renewcommand{\headrulewidth}{1pt}
%%%%%%%%%%%%%%%%%%%%%%%%%%%%%%%%%%%%%%%%

%%% Footer %%%%%%%%%%%%%%%%%%%%%%%%%%%%%
%\cfoot{blablablabla}
%\renewcommand{\footrulewidth}{0.4pt}
%%%%%%%%%%%%%%%%%%%%%%%%%%%%%%%%%%%%%%%%
%\date{} % Activate to display a given date or no date (if empty),
         % otherwise the current date is printed 

\begin{document}
%\maketitle
\bstctlcite{BSTcontrol} % IEEEtran.bst controls enabled


%----------------------------------------------------------------------------------------
%	TITLE PAGE
%----------------------------------------------------------------------------------------

\begin{titlepage} % Suppresses displaying the page number on the title page and the subsequent page counts as page 1
	\newcommand{\HRule}{\rule{\linewidth}{0.5mm}} % Defines a new command for horizontal lines, change thickness here
	
	\center % Centre everything on the page
	
	%------------------------------------------------
	%	Headings
	%------------------------------------------------
	
	\begin{figure}
   		\centering
    	\includegraphics[scale=1]{kthLogo.png}
	\end{figure}
	
	\textsc{\LARGE KTH Management of Projects}\\[1cm] % Main heading such as the name of your university/college
	
	\textsc{\Large EH2760, HT 2018}\\[0.5cm] % Major heading such as course name
	
	\textsc{\Large FINAL REPORT}\\[0.5cm] % Major heading as well
	
	%------------------------------------------------
	%	Title
	%------------------------------------------------
	
	\HRule\\[0.4cm]
	
	{\huge\bfseries ESS-NW/ESS-CAR}\\[0.4cm] % Title of your document
	
	\HRule\\[1.5cm]	
	
	%------------------------------------------------
	%	Author(s)
	%------------------------------------------------
	
	\begin{minipage}{0.4\textwidth}
		\begin{flushleft}
			\large
                        \textsc{Jonas Ekman}
			\\
			\textsc{Yini Gao}
                        \\
                        \textsc{Jacob Kimblad}
		\end{flushleft}
	\end{minipage}
	~
	\begin{minipage}{0.4\textwidth}
		\begin{flushright}
			\large
                        \textsc{Leon Fernandez}
			\\
			\textsc{Fredrik Hyyrynen}
                        \\
                        \textsc{Yifan Ruan}
		\end{flushright}
	\end{minipage}
	
	% If you don't want a supervisor, uncomment the two lines below
        % and comment the code above
	%{\large\textit{Author}}\\
	%John \textsc{Smith} % Your name
	\vskip 8cm
	\begin{figure}[H]
   		\centering
    	\includegraphics[scale=0.4]{funLogo.png}
	\end{figure}
	%------------------------------------------------
	%	Date
	%------------------------------------------------
	
	\vfill\vfill\vfill % Position the date 3/4 down the remaining page
	
	{\large\today} % Date, change the \today to a set date if you want
                       % to be precise
	
	%------------------------------------------------
	%	Logo
	%------------------------------------------------
	
	%\vfill\vfill
	%\includegraphics[width=0.2\textwidth]{placeholder.jpg}\\[1cm]
        % Include a department/university logo
        % - this will require the graphicx package
	 
	%-------------------------------------------------------------------
	
	\vfill % Push the date up 1/4 of the remaining page
	
\end{titlepage}

%-------------------------------------------------------------------------

\clearpage

\section{General Summary}
A prototype for an autonomous car has been built. As requested, the car is able to
do some basic self-monitoring and uses ethernet with a network controller for on-board
communication. Unfortunately, it is not able to consistently track a flag, which was
another requirement.

Due to integration issues, a lot more time than planned had to be spent during the
last weeks of the project. The budget was exceeded by roughly 74\%.

The integration issues consisted mostly of problems with the interposes communication
between different software tasks. During integration, it was also found that ordering
PCBs via KTH was not possible and that the PCB machines available to the group were broken.
The interposes communication and the PCB issues were the main causes of delay in the
project.

\section{Follow-up of objectives}
\subsection{Goals}
\begin{enumerate}
\item The computer vision system of the car does not have the required performance to
properly follow a green circle held in front of the car.

\item The hardware has been extended to include one unique processor for each sensor or actuator
node and the car has all the functionality it needs to keep a distance of 30 cm to an
object in front.

\item The car has been made failsafe to the extent that it can detect if any of the major
nodes are missing. Most importantly, the node that controls the motor can by itself
stop the car within the required time, even if it gets isolated from the other nodes.
No battery level reading has been implemented.

\item The software running on the controller node checks the availability of all subsystem
and does not start if any of them are missing.

\item The physical layer has been set in a topology that allows the SDN-system
to route packets to and from all nodes.

\item The SDN system consists of a SDN-controller that runs on a Raspberry Pi and Zodiac FX OpenFlow
switches.

\item There is only one route between any two given nodes that are connected as hosts to the
SDN system. As long as the SDN system is activated, packets are always take this route.

\item The SDN controller does not reassign network resources during runtime.
\end{enumerate}

\subsection{Action Plan}
Goals 1, 3 and 8 cannot be considered completed. No action will be taken for goal 3 since
the current situation was deemed acceptable by the stakeholders. The cause of failure for
goal 1 and 8 have been identified and the shareholders have been notified. The action taken
for these failures is to hand over a well-documented repository to the stakeholders so that
they can have another group of students solve these issues in a future course round.

\section{Lessons learned and suggestions for improvement}
The experience with the project has, in general, been good. Most members felt that they
got to work with relevant topics and try out new ideas. Furthermore, all members got to try out
professional engineering tools and methods that will be useful in the future. Some examples
include cmake, Doxygen, Slack, Fusion 360, Eagle and PyChar.

However, the group had quite a few experiences with inefficient meetings and throughout most
of the project there was a consensus that more people were required. Although, an effort was
made to implement a Scrum system using the Trello platform as well as a physical whiteboard, it
did not work out in the end. Also, the integration of all the software subsystems got
delayed, partly due to difficulties with both making and ordering a PCB.

Despite the difficulties with ordering a PCB and the inefficient meetings, most of the
communication with suppliers and stakeholders worked smoothly. Lastly, even though not
all goals were fulfilled, the project members felt that they made the right choices in
prioritizing what would be the final deliverables.

\subsection{Lessons learned}
\begin{itemize}
    \item To improve the time utilization, it would have been good to have a project
          manager who does little or no development.
    \item To improve the Scrum system, only one board should have been employed instead
          of using both a whiteboard and a digital Scrum board.
    \item To improve the meeting efficiency, a more formal meeting structure should have been
          employed.
    \item More flexibility when it comes to restating the goals. For instance, the goals should
          have been restated when it at one point was discovered that previous work would have to
          be redone.
    \item Write more detailed technical requirements that can help facilitate the task breakdown.
\end{itemize}

\section{Summary of time and resource plans}
The final versions of the time and resource plans can be seen in Figure~\ref{fig:resource_plan}
and~\ref{fig:timeplan}. All tasks up until mid-november were dealt with in time. After that,
a lot of tasks hade to be postponed and the order of some of the tasks had to be changed as well.
This was mostly due to integration problems with the different software subsystems that had
been developed. Something that was missing in the time plan that could have mitigated this issue
was to set aside a few days for implementing the interposes communication between software tasks,
as this was the main reason for the integration problems.

Another reason for the delay was issues with ordering PCBs and broken PCB
prototype machines. Unfortunately there was nothing in the risk analysis that reflected
this specific issue, which could probably have been mitigated with some simple proactive
measures like hand soldering prototype cards early on in the project.

\textit{Note: Milestones that were never met are denoted by a red semicircle in
Figure~\ref{fig:timeplan}.}

\section{Final comment/other}
This section can include the project manager's sub-project managers' own views and comments on the project, for example: valid documentation, work practices, project management and more.


\section{Attachments}
Final resource plan
Final milestone char (or Gantt Chart if used)


\clearpage

\bibliographystyle{IEEEtran}

\bibliography{reference}
\clearpage
%-------------------------------------------------------------------------
\clearpage

\appendix
\section{EVM}
\begin{figure}[h]
     \centering
     \includegraphics[scale=0.8]{evm.pdf}
     \caption{Earned Value Management analysis.}
     \label{fig:evm}
\end{figure}

\section{Time Plan}
\begin{sidewaysfigure}
     \centering
     \includegraphics[scale=0.6]{timeplan.pdf}
     \caption{Time plan for the project.}
     \label{fig:timeplan}
\end{sidewaysfigure}

\section{Resource Plan}
\begin{sidewaysfigure}
     \centering
     \includegraphics[scale=0.8]{resource_plan.pdf}
     \caption{Resource plan for the project.}
     \label{fig:resource_plan}
 \end{sidewaysfigure}
\end{document}
